%
%  Vincent Yannello
%
\documentclass[12pt,fullpage]{article}
\usepackage{fullpage}
\usepackage{amsmath}
\DeclareMathOperator{\erf}{erf}
\usepackage{psfrag}                                          % LaTeX graphics tool
\usepackage{pslatex}                                         % avoids the default cmr font
\usepackage{graphicx}                                        % graphics package 
\usepackage{epsfig}                                          % figures
\usepackage{hyperref}
\usepackage{color}

\begin{document}

\noindent
{\bf Doubly noncentral t distribution} (from \color{blue}\url{http://www.math.wm.edu/~leemis/chart/UDR/UDR.html}\color{black})

\noindent
The shorthand $X \sim {\rm t''}(n,\, \delta,\, \gamma)$ is used to indicate that the
random variable $X$ has the doubly noncentral t distribution with positive integer parameter $n$ and positive noncentrality parameters $\delta$, $\gamma$.
A doubly noncentral t random variable $X$ with parameters $n$, $\delta$, and $\gamma$ is defined by the transformation
$$
t''(n, \delta, \gamma) = \frac{U + \delta}{\chi'(n, \gamma)/\sqrt{n}}
$$
with $U$ being a $N(0,1)$ random variable and $\chi'(n, \gamma)$ being distributed as a mixture of $\chi (n + 2j)$ distributions
in proportions $e^{-\gamma/2}(\gamma/2)^j/j!$, $j = 0, \, 1, \, 2, \, \ldots \,$.

\vspace{0.1in}
\noindent
Alternative representations exist, and can be found along with additional information
on page 533 of Johnson, N.L., Kotz, S., and Balakrishnan, N. (1994),
``Continuous Univariate Distribuions'' (Vol. I, 2nd ed.), New York: Wiley.
See also
Krishnan, Marakatha (1968), ``Series Representation of the Doubly Noncentral $t$-Distribution,'' {\it Journal of the American Statistical Association},
Volume 63, Number 323, pp.\ 1004--1012.

\vspace{0.1in}

\noindent

\end{document}
