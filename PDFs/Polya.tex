%
%  Vincent Yannello
%
\documentclass[12pt,fullpage]{article}
\usepackage{fullpage}
\usepackage{amsmath}
\DeclareMathOperator{\erf}{erf}
\usepackage{psfrag}                                          % LaTeX graphics tool
\usepackage{pslatex}                                         % avoids the default cmr font
\usepackage{graphicx}                                        % graphics package 
\usepackage{epsfig}                                          % figures
\usepackage{hyperref}
\usepackage{color}

\begin{document}

\noindent
{\bf Polya distribution} (from \color{blue}\url{http://www.math.wm.edu/~leemis/chart/UDR/UDR.html}\color{black})

\noindent
The shorthand $X \sim {\rm Polya}(n,\, p,\, \beta)$ is used to indicate that the
random variable $X$ has the Polya distribution with parameters $n$, $p$, and $\beta$.
A Polya random variable $X$ with parameters $n$, $p$, and $\beta$ has probability mass function 
$$
f(x) =  \frac{\displaystyle {n \choose x} \prod_{j = 0} ^ {x - 1}(p + j \kern 0.08 em \beta) \displaystyle\prod_{k = 0} ^ {n - x - 1}(1 - p + k \kern 0.08 em \beta)}
{\displaystyle \prod_{i = 0} ^ {n - 1} (1 + i \kern 0.08 em \beta)}
 \qquad \qquad x = 0, 1, 2,  \ldots, \, n,
$$
for all $n=1, 2, \ldots,$ $0<p<1$, and $\beta>0$.

\vspace{0.05in}

\noindent
The cumulative distribution, survivor function, hazard function, cumulative hazard 
function, and inverse distribution function, moment generating function, and characteristic function
on the support of $X$ are mathematically intractable.

\vspace{0.05in}

\noindent
The population mean of $X$ is
$$
E[X] = -\frac{\sin \left( {\frac {\pi \, \left( -1 + \beta + p \right) }{\beta}}
 \right) p \kern 0.08 em n \sin \left( {\frac {\pi \, \left( n \kern 0.08 em \beta + 1 \right) }{\beta}
} \right)}{\sin \left( {\frac {\pi } {\beta}} \right)\sin \left( {\frac {\pi \, \left( n \kern 0.08 em \beta - \beta - p + 1
 \right) } {\beta}} \right)}.
$$
\vspace{0.1in}

\noindent
{\bf APPL verification:}
The APPL statements
\begin{verbatim}
X := [[x -> binomial(n, x) * product(p + j * beta, j = 0 .. x - 1) *
      product(1 - p + k * beta, k = 0 .. n - x - 1) /
      (product(1 + i * beta, i = 0 .. n - 1))],  
      [0 .. n], ["Discrete", "PDF"]];
Mean(X);
Variance(X);
Skewness(X);
MGF(X);
\end{verbatim}
return the population mean, variance, skewness, and moment generating function.
\end{document}
